\documentclass[12pt]{article}

% Chargement des packages nécessaires
\usepackage[utf8]{inputenc}
\usepackage[T1]{fontenc}
\usepackage[francais]{babel}
\usepackage{graphicx}




\title{Architecture des Ordinateurs : Rapport de Projet}

\author{Licence 2 d'Informatique\\ 
  Architecture des Ordinateurs\\
  Projet de Borde Antoine et Clement Nerestand}




\begin{document}

\maketitle

\newpage

\tableofcontents

\newpage

\section*{Introduction}

\paragraph{} Au travers de ce rapport nous allons vous presenter le projet
 de l'UE architecture des ordinateurs ainsi que le travail que nous avons 
 effectué dessus.

\paragraph{} Ce projet a pour but d'étendre notre compréhension et notre 
maitrise du simulateur d'architecture y86 qui est une version simplifié
du x86 normes des processeurs actuels. Pour cela nous allons chercher à 
étendre le jeu d'instruction disponible en y86.

\paragraph{} Le projet ce présente sous forme de trois excercises
articulés autour des trois etapes
necessaire a l'ajout d'une nouvelle instruction à savoir la liberation
des icodes, rendre possible l'utilisation de plusieurs instructions en
une et pour finir l'implementation des instructions à rajouter.

\paragraph{} En effet le code d'une instruction commance toujours par le
 icode qui n'est rien d'autre que le numerau de l'instruction et qui
 donc est parfaitement indispensable. Hors il se trouve que dans le simulateur
 y86 tout les icodes sont deja utilisés par une instruction. Il va donc nous
 falloir factoriser des instructions en une afin de liberer des icodes.

\paragraph{} Pour ce qui est de permettre a notre simulateur d'éffectuer
plusieur instruction en une nous devront trouver une astuce nous permettant
d'injecter plusieur instruction les une a la suite des autre chacune ayant un
comportement different. Nous allons pour cela jouer avec le ifun qui ce situe
juste aprés le icode et qui permet de différencier plusieur instruction possedant le meme icode.

\paragraph{} Quand a l'implementation a proprement parler de nos instruction il
sagira de jouer avec le simulateur afin de bien gerer le comportement de nos instruction.

\newpage


\section{Exercice 1 : De la place dans les opcodes}
\subsection{Factorisation de ALU et ALUI}



\end{document}
